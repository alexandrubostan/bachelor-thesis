%!TEX root = ../thesis.tex

\chapter{Benefits}

This chapter outlines the primary benefits associated with integrating commercial smart devices with the ThingsBoard platform. These advantages are relevant from both technical and user-centric perspectives, particularly for developers, system integrators, and organizations seeking to improve the flexibility, longevity, and control of their IoT deployments.

\section{Repurposing Devices}

One of the most compelling motivations for integrating smart devices with an open-source IoT platform such as ThingsBoard is the ability to repurpose existing hardware. Commercial smart devices, including those manufactured under the Tuya ecosystem, are often tied to proprietary applications and cloud services. This dependence may limit their utility in custom or enterprise-grade IoT projects, where greater flexibility is required.

By enabling communication between Tuya-based devices and ThingsBoard, it becomes possible to decouple the hardware from its original ecosystem. This decoupling allows the devices to serve new functions beyond their initial consumer-oriented use cases. For instance, a smart temperature sensor initially used in a residential environment can be integrated into an industrial monitoring system, or a Wi-Fi-enabled smart plug can be incorporated into an energy optimization platform.

Repurposing not only reduces the need for additional hardware investment but also contributes to sustainability by extending the useful function of existing devices. This aligns with broader trends in circular economy practices and resource efficiency within the IoT sector.

\section{Extending Product Lifespan}

The integration of commercial smart devices with open-source platforms can significantly extend their operational lifespan. Many proprietary IoT ecosystems have limited long-term support, and devices may become obsolete when cloud services are discontinued or mobile applications are deprecated. This can lead to premature device obsolescence, even when the hardware remains fully functional.

By migrating device control and data management to a platform like ThingsBoard, users can maintain full operational control of their devices regardless of vendor support. This is particularly beneficial in scenarios where long-term device availability is critical, such as industrial deployments, research installations, or rural infrastructure projects.

In addition, ThingsBoard's modular and scalable architecture allows for continuous system upgrades without affecting device compatibility, thereby future-proofing the IoT ecosystem and protecting investments in hardware over time.

\section{Privacy}

Another substantial benefit of integrating smart devices with ThingsBoard is the increased control over user data and privacy. Many commercial IoT solutions rely heavily on cloud infrastructure owned and operated by third-party vendors. As a result, users often have limited visibility into how their data is stored, processed, or shared. This raises concerns related to data privacy, security, and regulatory compliance, particularly in sensitive domains such as healthcare, home security, and industrial operations.

In contrast, ThingsBoard can be deployed on-premises or in a self-hosted cloud environment, giving users full control over data flow and storage. This enables compliance with data protection regulations such as the General Data Protection Regulation (GDPR) and allows for the implementation of organization-specific security policies.

Moreover, data ownership remains with the user or the organization operating the ThingsBoard instance. This autonomy enhances transparency and trust, especially in applications where data sensitivity is a key consideration. The platform also allows for fine-grained access control and audit logging, further strengthening the security posture of the integrated system.
